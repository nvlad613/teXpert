\documentclass[14pt]{extarticle} % Set the font size to 14pt
\setlength\parindent{1.25cm} % Set the parindent to 1.25 cm
% Packages
\usepackage[utf8]{inputenc} % Set the UTF-8 coding
\usepackage[russian]{babel} % Set support of Russian language
\usepackage{amsmath} % Set support of Mathematical formulas
\usepackage{amssymb} % Set support for extra symbols
\usepackage[width=14cm, left=3cm]{geometry} % Set support of page dimensions
\usepackage{fontspec} % Set for using system fonts
\usepackage{ragged2e} % Set for advanced justification options
\usepackage{setspace} % Set for line spacing
\usepackage{titlesec} % Set for title settings
\usepackage{indentfirst} % Set for indent first
\usepackage{graphicx} % Set for image support
\usepackage[labelformat=simple]{caption} % Set for using caption
\usepackage[labelsep=endash]{caption} % Set the separator
\usepackage[hidelinks]{hyperref} % Set for using hyperlinks (hide mode)
\usepackage{enumitem} % Set for creating books list
\usepackage{booktabs} % Set for using modern tables
\setmainfont{Times New Roman} % Setting Times New Roman as main font
% Page dimensions
\geometry{
    a4paper, % Set the paper size
    left=30mm, % Set the left margin
    right=15mm, % Set the right margin
    top=20mm, % Set the top margin
    bottom=20mm % Set the bottom margin
}

% Change figure to "Рисунок"
\addto\captionsrussian{\renewcommand{\figurename}{Рисунок}}

% Change sections

% Change ToC title
\addto\captionsrussian{
  \renewcommand{\contentsname}%
    {СОДЕРЖАНИЕ}%
}

% Title elements settings
\titleformat{\section}
 {\normalfont\bfseries\Large\raggedright\centering}{\thesection}{1em}{}

\titleformat{\subsection}
  {\normalfont\bfseries\raggedright}{\hspace{1.25cm}\thesubsection}{1em}{}

\titleformat{\subsubsection}
  {\normalfont\bfseries\raggedright}{\hspace{1.25cm}\thesubsubsection}{1em}{}

% New section on new page settings
\newcommand{\newsection}[1]{
    \clearpage
    \section{#1}
}

\newcommand{\newsectionstar}[1]{%
    \clearpage
    \section*{#1} % Create an unnumbered section
    \addcontentsline{toc}{section}{#1} % Add it to the table of contents
}

% Title information
\title{Шаблон ВКР} % Document title in Russian
\author{Дробыш Дмитрий Александрович} % Your name
\date{\today} % Date

\begin{document}
\pagestyle{plain} % Set the page style to plain (gives page numbers)
\onehalfspacing % Set the line spacing to 1.5
\justifying % Ensure text is justified

\maketitle
\newpage
\tableofcontents

% Список сокращений и условных обозначений
\newsectionstar{СПИСОК СОКРАЩЕНИЙ И УСЛОВНЫХ ОБОЗНАЧЕНИЙ}

РН — Ракета-носитель\\

НПО — Научно-Производственное Объединение\\

ГКЧП — Государственный Комитет по Чрезвычайному Положению\\

NASA — Национальное управление по аэронавтике и исследованию космического пространства (США)\\

РКК — Ракетно-космическая корпорация\\

КБ — Конструкторское Бюро\\


% Термины и определения
\newsectionstar{ТЕРМИНЫ И ОПРЕДЕЛЕНИЯ}
Ракета-носитель (РН) — это специальный летательный аппарат, предназначенный для вывода полезной нагрузки (например, спутников, научных аппаратов или экипажей) на орбиту Земли или в межпланетное пространство. Ракетные носители могут быть одноразовыми или многоразовыми и обычно состоят из нескольких ступеней, каждая из которых оснащена ракетными двигателями.\\

Научно-Производственное Объединение (НПО) — это организация, объединяющая научные исследования и производственные мощности для разработки и производства высокотехнологичной продукции. В контексте ракетостроения НПО часто занимаются разработкой новых технологий, проектированием ракет и космических аппаратов, а также их серийным производством.\\

Государственный Комитет по Чрезвычайному Положению (ГКЧП) — это орган власти, созданный в СССР в августе 1991 года для управления страной в условиях чрезвычайной ситуации. Его деятельность была связана с попыткой сохранить советскую власть и предотвратить распад СССР. ГКЧП стал известен благодаря своему неудачному путчу, который привел к усилению демократических движений и окончательному распаду Советского Союза.\\

NASA (National Aeronautics and Space Administration) — это федеральное агентство правительства США, ответственное за гражданскую аэрокосмическую программу и научные исследования в области аэронавтики и космоса. Основанное в 1958 году, NASA проводит миссии по исследованию космоса, включая пилотируемые полеты, запуск спутников и научные исследования на других планетах.

Ракетно-космическая корпорация (РКК) — это организация, занимающаяся разработкой, производством и эксплуатацией ракетных систем и космических аппаратов. В России одной из наиболее известных РКК является Ракетно-космическая корпорация "Энергия", которая разрабатывает космические технологии и принимает участие в международных космических проектах.\\

Конструкторское бюро (КБ) — это специализированное подразделение или организация, занимающаяся проектированием и разработкой новых технических решений, изделий и технологий. В области ракетостроения КБ разрабатывают проекты ракетных систем, космических аппаратов и других сложных технических устройств, проводя исследования и испытания для подтверждения их работоспособности.

% Введение
\newsectionstar{ВВЕДЕНИЕ}
История ракетостроения начинается с древних времен, когда люди использовали простые устройства, такие как фейерверки и ракеты на порохе. С течением времени эти технологии развивались, что привело к созданию более сложных ракетных систем. В XX веке ракетостроение стало ключевым элементом в освоении космоса, что открыло новые горизонты для науки и техники. В данной работе будет рассмотрена эволюция ракетостроения, начиная с первых экспериментов и заканчивая современными достижениями.\\


% Основной раздел
\newsection{ОСНОВНОЙ РАЗДЕЛ}

\subsection{ПЕРВЫЕ ШАГИ В РАКЕТОСТРОЕНИИ}
Исторически первой известной ракетой считается китайская ракета "фейерверк", использовавшаяся в военных целях в XII веке. В Европе эксперименты с ракетами начали проводиться в XVIII веке. Однако настоящая революция в ракетостроении произошла в начале XX века благодаря работам таких ученых, как Константин Циолковский и Роберт Годдард.

\subsection{СОВРЕМЕННЫЕ ДОСТИЖЕНИЯ}
Сегодня ракетостроение продолжает развиваться с небывалой скоростью. Новые технологии, такие как многоразовые ракеты, позволяют значительно снизить стоимость космических запусков. Компании, такие как SpaceX и Blue Origin, активно работают над инновациями, которые могут изменить будущее космических исследований.

SpaceX Starship изображен на рисунке~\ref{fig:starship}.
\begin{figure}[h]
    \centering
    \includegraphics[width=0.3\textwidth]{starship.png}
    \caption{SpaceX Starship}
    \label{fig:starship}
\end{figure}

Blue Origin New Glenn изображен на рисунке~\ref{fig:newglenn}.
\begin{figure}[h]
    \centering
    \includegraphics[width=0.3\textwidth]{newglenn.png}
    \caption{Blue Origin New Glenn}
    \label{fig:newglenn}
\end{figure}

% Второй раздел
\newsection{ТЕХНИЧЕСКИЕ ХАРАКТЕРИСТИКИ SPACEX STARSHIP}
% Пример таблицы
В данном разделе мы представим таблицу. Как показано в таблице~\ref{tab:starship}, это пример оформления таблицы.

% Пример таблицы
\begin{table}[h]
\centering % Центрирование таблицы
\captionsetup{justification=raggedright, singlelinecheck=false}
\caption{Характеристики SpaceX Starship} % Заголовок таблицы
\label{tab:starship}
\begin{tabular}{|l|l|}
\hline
\textbf{Категория} & \textbf{Описание}                              \\ \hline
Название           & Starship                                       \\ \hline
Разработчик        & SpaceX                                         \\ \hline
Длина              & Около 50 м                                     \\ \hline
Диаметр            & 9 м                                            \\ \hline
Полезная нагрузка  & До 100 т в LEO                                 \\ \hline
Двигатели          & 6 Raptor (3 вакуумных, 3 атмосферных)          \\ \hline
Топливо            & Жидкий метан и кислород                        \\ \hline
Первый полет       & Ожидается в ближайшие годы                     \\ \hline
Основные цели      & Пилотируемые миссии на Марс, исследования Луны \\ \hline
Текущий статус     & В стадии испытаний                             \\ \hline
\end{tabular}
\end{table}
Скорость можно найти как:

\begin{equation}
	v = I \cdot g \cdot ln((m / m_f)) \label{eq:speed},
\end{equation}

где: $v$ - скорость ракеты, м/v; $I$ — удельный импульс двигателя, с; $g$ —  ускорение свободного падения на поверхности Земли, 9.81 м/с²; $m$ — начальная масса ракеты (включая топливо), кг; $m_f$ — конечная масса ракеты (без топлива), кг;

% Заключение
\newsectionstar{ЗАКЛЮЧЕНИЕ}
История ракетостроения — это история непрерывного поиска и инноваций. От первых пороховых ракет до многоразовых космических кораблей — каждый этап развития был важен для достижения современных успехов в области космических исследований. Будущее ракетостроения обещает быть еще более захватывающим, открывая новые возможности для человечества.

% Список источников
\newsectionstar{СПИСОК ИСПОЛЬЗОВАННЫХ ИСТОЧНИКОВ}
\begin{enumerate}[label=\arabic*., % Correct label syntax
                  left=1.25cm    % Correct left margin syntax
		  ]
    \RaggedRight

    \item  Циолковский К.Э. Исследование мировых пространств реактивными приборами // Вестник воздухоплавания. 1911, № 19-22, 1912, №2, 3, 5-7, 9.

    \item NASA: "History of Rocketry and Space Travel".  URL: \\
    \url{https://ui.adsabs.harvard.edu/abs/1975hrst.book.....V/abstract} (дата обращения 26.02.2025).
\end{enumerate}

% Приложение
\newsectionstar{ПРИЛОЖЕНИЕ A}
\end{document}