\documentclass[a4paper,14pt]{extarticle}

% -----------------------------
% Пакеты для кириллицы и русской типографики
\usepackage[utf8]{inputenc}
\usepackage[T2A]{fontenc}
\usepackage[russian]{babel}

% Параметры страницы
\usepackage{geometry}
\geometry{a4paper, left=45mm, right=15mm, top=20mm, bottom=20mm}
\usepackage{setspace}
\onehalfspacing  % полуторный интервал

% Шрифт TeX Gyre Termes (Times-подобный с более заметной жирностью)
\usepackage{tgtermes}

% Управление заголовками (section, subsection и т.д.)
\usepackage{titlesec}
\newcommand{\sectionfont}{\normalfont\fontsize{16pt}{14pt}\selectfont\bfseries\centering\MakeUppercase}
\newcommand{\subsectionfont}{\normalfont\fontsize{16pt}{14pt}\selectfont\bfseries}
\newcommand{\subsubsectionfont}{\normalfont\fontsize{16pt}{14pt}\selectfont\bfseries}

\titleformat{name=\section,numberless}{\sectionfont}{}{0pt}{}   % Ненумерованные (напр. Введение)
\titleformat{name=\section}{\sectionfont}{\thesection}{1em}{}   % Нумерованные секции
\titleformat{\subsection}{\subsectionfont}{\thesubsection}{1em}{} 
\titleformat{\subsubsection}{\subsubsectionfont}{\thesubsubsection}{1em}{} 

% Убираем автоматический заголовок "Список литературы" в thebibliography
% (Russian babel в классах иногда вставляет "СПИСОК ЛИТЕРАТУРЫ", поэтому переопределяем \bibname)
\makeatletter
\renewcommand{\bibname}{}  % Пустое, чтобы в PDF не выводилось автоматически 
\makeatother

% Нумерация страниц
\usepackage{fancyhdr}
\pagestyle{fancy}
\fancyhf{}
\rfoot{\thepage}
\renewcommand{\headrulewidth}{0pt}

% Настройка абзацев
\setlength{\parindent}{1.25cm}

% Настройка оглавления
\usepackage{tocloft}
\renewcommand{\cftsecleader}{\cftdotfill{\cftdotsep}}
% Заголовок "СОДЕРЖАНИЕ" по центру, капсом, жирным, 14pt
\renewcommand{\cfttoctitlefont}{\hfill\normalfont\fontsize{14pt}{14pt}\selectfont\bfseries\MakeUppercase}
\renewcommand{\cftaftertoctitle}{\hfill}

\begin{document}

% Оглавление
\tableofcontents
\newpage

% Введение (ненумерованный раздел)
\section*{Введение}
\addcontentsline{toc}{section}{Введение}


\section{Аналитический обзор}
\subsection{Методы представления трехмерных объектов в машинном обучении}
\subsection{Методы уменьшения размерности трехмерных объектов}
\subsection{Альтернативные способы представления трехмерных объектов}

\section{Исследование технологии T-сплайнов и их реализации}
\subsection{Теоретические основы T-сплайнов и их свойства}
\subsection{Sub-D как практическая реализация T-сплайнов}
\subsection{Технология преобразования мешей в Sub-D и обратно}

\section{Разработка метода снижения размерности трехмерных моделей}
\subsection{Выбор датасета для работы}
\subsection{Используемые алгоритмы преобразования меша в Sub-D}
\subsection{Оценка изменения размерности данных}
\subsection{Выбор метрик для оценки эффективности преобразования}

\section{Разработка 3D-CNN для работы с уменьшенными моделями}
\subsection{Выбор архитектуры 3D-CNN}
\subsection{Подготовка данных на основе преобразованных 3D-моделей}
\subsection{Реализация и обучение нейросети}

\section{Экспериментальное исследование}
\subsection{Проведение тестирования разработанной методики}
\subsection{Сравнение с другими методами представления 3D-данных}
\subsection{Анализ точности, производительности и эффективности подхода}

\section{Выводы и перспективы дальнейших исследований}
\subsection{Итоговые результаты работы}

% ------------------------------
% Список использованных источников
% ------------------------------
% Ненумерованный раздел "Список использованных источников"
\section*{Список использованных источников}
\addcontentsline{toc}{section}{Список использованных источников}



\newpage
% Приложение (ненумерованный)
\section*{Приложение А}
\addcontentsline{toc}{section}{Приложение А}
В данном приложении может быть приведён пример листинга кода, дополнительная графика или большие таблицы.
\end{document}
